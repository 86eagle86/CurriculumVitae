\documentclass[pdftex,a4paper,10pt,twoside,titlepage,italian]{article}
\usepackage[italian]{babel}
\usepackage[utf8]{inputenc}
\usepackage{fancyhdr,graphicx}
\usepackage[hmargin=2cm,vmargin=1.5cm,a4paper]{geometry}
\usepackage{hyperref}
\usepackage{xcolor}
\usepackage{multirow}
\usepackage{longtable}
\pagestyle{fancy}
\lhead{\scshape Curriculum Vitae}
\rhead{\itshape Mauro Donadeo}
\rfoot{\footnotesize pag. \thepage}
\cfoot{}
\lfoot{{\footnotesize Aggiornato: Dicembre 2013}}
\renewcommand{\headrulewidth}{.5pt}
\renewcommand{\footrulewidth}{.5pt}
%\renewcommand{\LettrineFontHook}{\color[gray]{0.5}}
%\renewcommand*\familydefault{\sfdefault}
\renewcommand*\labelitemi{$\textcolor{black!30}{\bullet}$}
%\definecolor{listings-comment-color}{RGB}{20,0,0}

\begin{document}
\begin{center}
\rule{.8 \textwidth}{1pt}\\[5pt]
\begin{minipage}{.55\textwidth}
	\LARGE\textbf{Mauro Donadeo}\\[16pt]
	\footnotesize Via Isonzo 136/7 \\ 
	35143 - Padova (PD)\\
	email: \texttt{mauro.donadeo@gmail.com}\\
	\footnotesize{data di nascita: 18 Gennaio 1986}\\
	\footnotesize {n. Tel: +39 346 784 6243}\\
	\footnotesize {Nazionalità: Italiana}\\
	\footnotesize {Stato civile: Celibe}\\
	\footnotesize {Skype: mauro.donadeo}
\end{minipage}
\begin{minipage}{.25\textwidth}
	\includegraphics[width=\textwidth]{io.jpg}
\end{minipage}
\rule{.8 \textwidth}{1pt}\\[5pt]
\end{center}
\section*{Esperienze Professionali}
\begin{longtable}[h]{l p{0.8\textwidth}}
\footnotesize{data (da - a)} & Ottobre 2013 - Oggi \\
\footnotesize{posizione} & Assistente ufficio IT presso E-Global Service Spa.\\
\footnotesize{attività principali}& Sviluppo di applicazioni e servizi Windows tramite Framework \textit{.Net} (3.5) e 
Microsoft Sql Server 2008. Creata una nuova barra delle applicazioni aziendali utilizzando la libreria grafica \textit{WPF}.\\
& \\
\footnotesize{data (da - a)} & Luglio 2012 - Luglio 2013 \\
\footnotesize{posizione} & Assegno Grant presso l'Università degli Studi di Padova - DEI.\\
\footnotesize{attività principali}& Creazione di un sistema di hand detection e hand gesture recognition.
Mediante acquisizione delle immagini con sistemi low cost (Microsoft Kinect). Sistema sviluppato interamente in 
\textit{C++} utilizzando le librerie \textit{OpenCV, OpenNI, VTK, PCL e Qt}.\\
 & \\
\footnotesize{data (da-a)} & Gennaio 2012 - Giugno 2012 \\
\footnotesize{posizione}& Contratto Co.Co.Co presso  l'Università degli Studi di Padova - DPG.\\
\footnotesize{attività principali} & Sviluppo di un sistema di riconoscimento gesture con immagini 
acquisite tramite MS Kinect. Sistema sviluppato 
 			in \textit{C++} utilizzando le librerie \textit{OpenCV e OpenNI}, animazioni riprodotte in \textit{Unity3D}. 
Questo sistema è stato sviluppato all'interno del progetto europeo CEEDs (http://ceeds-project.eu/).\\
& \\
\footnotesize{data (da - a)} & Marzo 2012 - Aprile 2012\\
\footnotesize{posizione} & Docente Corso C.A.R (Corso Avanzato per la Ricerca)\\
\footnotesize{attività principali} & Corso di 25 ore in cui sono stati spiegati i fondamenti del linguaggio di programmazione Java\\
& \\
\footnotesize{data (da - a)} & Novembre 2011 - Gennaio 2012 \\
\footnotesize{posizione} & Web Designer presso Simnumerica S.r.l.\\
\footnotesize{attività principali}& Sviluppo della pagina web dell'azienda. Moduli principalmente sviluppati: modulo di 
		iscrizione, news slider, e sistema di trouble ticketing. Creato modulo ad hoc per la sincronizzazione automatica degli
		utenti profilati sul sito web con il DB aziendale in cloud su Windows Azure\\
		& \\
\footnotesize{data (da - a)} & Settembre 2008 ad Agosto 2009 \\
\footnotesize{posizione} & Assistente informatico presso Università degli Studi di Padova: 
					Ufficio Certificazione Offerta Didattica - SID.\\
\footnotesize{attività principali} & Bonifica dati della base dati dell'Università, area didattica, prendendo come 
		modello i dati provenienti dal ministero.\\
		& Creazione di una piccola applicazione web per il controllo delle attività didattiche
		per la generazione del \textit{diploma supplement}.\\
& \\
\footnotesize{data (da - a)} & Dicembre 2007 - Agosto 2008 \\
\footnotesize{posizione} & Assistente informatico presso Regione del Veneto: Ufficio Tecnico e Coordinamento Reti.\\
\footnotesize{attività principali} & Progettazione e creazione del portale {\texttt overnetwork.it} basato sul {\textit cms Joomla}
		ed in particolare: creazione di un sistema di \textit{trouble ticketing} con conseguente 
		invio di notifiche tramite mail. Creazione di un nuovo sistema di \textit{profiling}
		degli utenti differente da quello previsto dalla piattaforma Joomla\\
& \\
\footnotesize{data (da - a)} & Settembre 2005 - Ottobre 2007 \\
\footnotesize{attività principali} & Portalettere per vari periodi presso i comuni di Cittadella (PD) e Campodarsego (PD).\\
%& \\
\end{longtable}

\section*{Istruzione}
\begin{itemize}
	\item Laurea Magistrale in Ingegneria Informatica (Università degli Studi di Padova, Ottobre 2011).
	Titolo tesi: {\itshape Creazione di un sistema di videoconferenza 3D basato sul sensore MS Kinect}
	\item Laurea Triennale in Ingegneria Informatica (Università degli Studi di Padova, Aprile 2008). 
	Titolo tesi: {\itshape Realizzazione di un simulatore elementare del sistema robotico Lego Nxt
	in Java}
	\item Diploma Capo-tecnico Informatico (I.T.I.S. A. Meucci di Casarano (LE)s, 2004)
\end{itemize}
\section*{Esperienze all'estero}
\begin{itemize}
	\item 4th CEEDs meeting Londra (Febbraio 2012).
	\item Erasmus presso FIB - Facultat d’Informàtica de Barcelona (a.a. 2009/2010)
\end{itemize}

\section*{Conoscenze informatiche:}
\begin{itemize}
	\item \textbf{S.O.:} Gnu/Linux conoscenza 	che va dall'installazione alla configurazione. Conoscenza anche dei vari 
	sistemi operativi Microsoft.
	\item \textbf{Programmazione: }:
	\begin{itemize}
	\item C/C++/C\#(base);
	\item Java
	\item Php MySql.
	\end{itemize}
	\item \textbf{Altro: }
	\begin{itemize}
	\item librerie per la {\itshape computer vision}: Directx10, OpenNI, OpenGL, OpenCV, Qt4;
	\item sviluppo software su sistemi di acquisizione stereo e MS Kinect;
	\item ambienti di sviluppo: NetBeans, Visual Studio 2010, QtCreator;
	\item Unity 3D Game Engine
	\item vim, Matlab, pacchetto Office, NXC, \LaTeX, GIT, SVN, SqlServer 2008.
	\end{itemize}
\end{itemize}
\section*{Conoscenze linguistiche:}
\begin{itemize}
	\item Madrelingua: \textbf{Italiano};
	\item altre lingue:
\end{itemize}
\begin{center}
	\begin{tabular}{|l|c|c|}
	\hline
	&\textbf{Inglese} & \textbf{Spagnolo}\\
	\hline
	Capacità di lettura & Buona & Eccellente\\
	\hline
	Capacità di scrittura & Discreta & Buona \\
	\hline
	Capacità di espressione orale & Buona & Eccellente \\
	\hline
	\end{tabular}
\end{center}
\section*{Pubblicazioni:}
\begin{itemize}
\item F. Dominio, M. Donadeo, G. Marin, P. Zanuttigh, G. M. Cortelazzo, ``\textit{Hand Gesture Recognition with Depth Data}'', Analysis and Retrieval of Tracked Events and Motion in Imagery Streams 2013, October 21, Barcelona, Spain.
\item G. Marin, M. Fraccaro, M. Donadeo, F. Dominio, P. Zanuttigh, ``\textit{Palm area detection for reliable hand gesture recognition}'', International Workshop on Multimedia Signal Processing 2013, Sept. 30 – Oct. 2, 2013, Pula (Sardinia), Italy.
\item F. Dominio, M. Donadeo, P. Zanuttigh, ``\textit{Combining multiple depth-based descriptors for hand gesture recognition}'',
accepted for Pattern Recognition Letters.
\item L. Nanni, A. Lumini, F. Dominio, M. Donadeo, P. Zanuttigh, ``\textit{Ensemble to improve gesture recognition}'',
accepted for International Journal of Automated Identification Technology.
\end{itemize}
\section*{Interessi personali: }
\begin{itemize}
	\item Informatica in generale, editing video e immagini;
	\item Sport: pratico nuoto e bicicletta da quest'anno sono iscritto ad una squadra di triathlon;
	\item Libri possibilmente in spagnolo per mantenere la conoscenza della lingua.
\end{itemize}
\vfill

\textit{Ai sensi del D. Lgs n. 196/2003 autorizzo al trattamento dei miei dati personali.}

\end{document}
